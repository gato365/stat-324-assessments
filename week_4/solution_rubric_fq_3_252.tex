% Options for packages loaded elsewhere
\PassOptionsToPackage{unicode}{hyperref}
\PassOptionsToPackage{hyphens}{url}
\PassOptionsToPackage{dvipsnames,svgnames,x11names}{xcolor}
%
\documentclass[
  letterpaper,
  DIV=11,
  numbers=noendperiod]{scrartcl}

\usepackage{amsmath,amssymb}
\usepackage{iftex}
\ifPDFTeX
  \usepackage[T1]{fontenc}
  \usepackage[utf8]{inputenc}
  \usepackage{textcomp} % provide euro and other symbols
\else % if luatex or xetex
  \usepackage{unicode-math}
  \defaultfontfeatures{Scale=MatchLowercase}
  \defaultfontfeatures[\rmfamily]{Ligatures=TeX,Scale=1}
\fi
\usepackage{lmodern}
\ifPDFTeX\else  
    % xetex/luatex font selection
\fi
% Use upquote if available, for straight quotes in verbatim environments
\IfFileExists{upquote.sty}{\usepackage{upquote}}{}
\IfFileExists{microtype.sty}{% use microtype if available
  \usepackage[]{microtype}
  \UseMicrotypeSet[protrusion]{basicmath} % disable protrusion for tt fonts
}{}
\makeatletter
\@ifundefined{KOMAClassName}{% if non-KOMA class
  \IfFileExists{parskip.sty}{%
    \usepackage{parskip}
  }{% else
    \setlength{\parindent}{0pt}
    \setlength{\parskip}{6pt plus 2pt minus 1pt}}
}{% if KOMA class
  \KOMAoptions{parskip=half}}
\makeatother
\usepackage{xcolor}
\setlength{\emergencystretch}{3em} % prevent overfull lines
\setcounter{secnumdepth}{-\maxdimen} % remove section numbering
% Make \paragraph and \subparagraph free-standing
\ifx\paragraph\undefined\else
  \let\oldparagraph\paragraph
  \renewcommand{\paragraph}[1]{\oldparagraph{#1}\mbox{}}
\fi
\ifx\subparagraph\undefined\else
  \let\oldsubparagraph\subparagraph
  \renewcommand{\subparagraph}[1]{\oldsubparagraph{#1}\mbox{}}
\fi


\providecommand{\tightlist}{%
  \setlength{\itemsep}{0pt}\setlength{\parskip}{0pt}}\usepackage{longtable,booktabs,array}
\usepackage{calc} % for calculating minipage widths
% Correct order of tables after \paragraph or \subparagraph
\usepackage{etoolbox}
\makeatletter
\patchcmd\longtable{\par}{\if@noskipsec\mbox{}\fi\par}{}{}
\makeatother
% Allow footnotes in longtable head/foot
\IfFileExists{footnotehyper.sty}{\usepackage{footnotehyper}}{\usepackage{footnote}}
\makesavenoteenv{longtable}
\usepackage{graphicx}
\makeatletter
\def\maxwidth{\ifdim\Gin@nat@width>\linewidth\linewidth\else\Gin@nat@width\fi}
\def\maxheight{\ifdim\Gin@nat@height>\textheight\textheight\else\Gin@nat@height\fi}
\makeatother
% Scale images if necessary, so that they will not overflow the page
% margins by default, and it is still possible to overwrite the defaults
% using explicit options in \includegraphics[width, height, ...]{}
\setkeys{Gin}{width=\maxwidth,height=\maxheight,keepaspectratio}
% Set default figure placement to htbp
\makeatletter
\def\fps@figure{htbp}
\makeatother

\KOMAoption{captions}{tableheading}
\makeatletter
\makeatother
\makeatletter
\makeatother
\makeatletter
\@ifpackageloaded{caption}{}{\usepackage{caption}}
\AtBeginDocument{%
\ifdefined\contentsname
  \renewcommand*\contentsname{Table of contents}
\else
  \newcommand\contentsname{Table of contents}
\fi
\ifdefined\listfigurename
  \renewcommand*\listfigurename{List of Figures}
\else
  \newcommand\listfigurename{List of Figures}
\fi
\ifdefined\listtablename
  \renewcommand*\listtablename{List of Tables}
\else
  \newcommand\listtablename{List of Tables}
\fi
\ifdefined\figurename
  \renewcommand*\figurename{Figure}
\else
  \newcommand\figurename{Figure}
\fi
\ifdefined\tablename
  \renewcommand*\tablename{Table}
\else
  \newcommand\tablename{Table}
\fi
}
\@ifpackageloaded{float}{}{\usepackage{float}}
\floatstyle{ruled}
\@ifundefined{c@chapter}{\newfloat{codelisting}{h}{lop}}{\newfloat{codelisting}{h}{lop}[chapter]}
\floatname{codelisting}{Listing}
\newcommand*\listoflistings{\listof{codelisting}{List of Listings}}
\makeatother
\makeatletter
\@ifpackageloaded{caption}{}{\usepackage{caption}}
\@ifpackageloaded{subcaption}{}{\usepackage{subcaption}}
\makeatother
\makeatletter
\@ifpackageloaded{tcolorbox}{}{\usepackage[skins,breakable]{tcolorbox}}
\makeatother
\makeatletter
\@ifundefined{shadecolor}{\definecolor{shadecolor}{rgb}{.97, .97, .97}}
\makeatother
\makeatletter
\makeatother
\makeatletter
\makeatother
\ifLuaTeX
  \usepackage{selnolig}  % disable illegal ligatures
\fi
\IfFileExists{bookmark.sty}{\usepackage{bookmark}}{\usepackage{hyperref}}
\IfFileExists{xurl.sty}{\usepackage{xurl}}{} % add URL line breaks if available
\urlstyle{same} % disable monospaced font for URLs
\hypersetup{
  pdftitle={Friday Quiz 3 Solutions - STAT 252},
  colorlinks=true,
  linkcolor={blue},
  filecolor={Maroon},
  citecolor={Blue},
  urlcolor={Blue},
  pdfcreator={LaTeX via pandoc}}

\title{Friday Quiz 3 Solutions - STAT 252}
\author{}
\date{}

\begin{document}
\maketitle
\ifdefined\Shaded\renewenvironment{Shaded}{\begin{tcolorbox}[frame hidden, enhanced, borderline west={3pt}{0pt}{shadecolor}, interior hidden, boxrule=0pt, sharp corners, breakable]}{\end{tcolorbox}}\fi

\hypertarget{question-10}{%
\subsection{Question 10}\label{question-10}}

Use the following information for question 10:

Panda Express aims to assess the potential success of a new line of
plant-based entrees compared to their classic menu options among
health-conscious Cal Poly students. To conduct this evaluation, they
select a sample of 1500 students at random. Among them, 1000
participants try the new plant-based entrees, while the remaining 500
are given the traditional menu items. After sampling the dishes,
participants are asked to indicate their satisfaction level with the
food they tried. Among those who tried the new plant-based entrees, 725
out of 1000 expressed satisfaction, while among those who tried the
traditional menu items, 300 out of 500 reported satisfaction. Panda
Express seeks to determine if there is a statistically significant
difference in customer satisfaction among Cal Poly students between
these two menu options.

\hypertarget{sample-proportions-phat_1-0.725-phat_2-0.600}{%
\subsection{Sample Proportions: phat\_1 = 0.725; phat\_2 =
0.600}\label{sample-proportions-phat_1-0.725-phat_2-0.600}}

\hypertarget{standard-error-0.02547875}{%
\subsection{Standard Error =
0.02547875}\label{standard-error-0.02547875}}

Calculate the test statistic. Interpret this value in the context of the
problem.

Here's a \textbf{full solution and 4-point rubric} in \textbf{Markdown
format} for the question asking students to compute and interpret the
\textbf{test statistic} for comparing two proportions.

\begin{center}\rule{0.5\linewidth}{0.5pt}\end{center}

\hypertarget{solution-full-credit-44}{%
\subsection{\texorpdfstring{\textbf{Solution (Full Credit --
4/4)}}{Solution (Full Credit -- 4/4)}}\label{solution-full-credit-44}}

\hypertarget{step-1-compute-the-test-statistic}{%
\subsubsection{\texorpdfstring{\textbf{Step 1: Compute the Test
Statistic}}{Step 1: Compute the Test Statistic}}\label{step-1-compute-the-test-statistic}}

We use the formula for a two-sample z-test for proportions:

\[
\text{TS} = \frac{\hat{p}_1 - \hat{p}_2}{SE}
\]

\[
\text{TS} = \frac{0.725 - 0.600}{0.02547875} \approx \frac{0.125}{0.02547875} \approx 4.91
\]

\hypertarget{step-2-interpretation}{%
\subsubsection{\texorpdfstring{\textbf{Step 2:
Interpretation}}{Step 2: Interpretation}}\label{step-2-interpretation}}

A test statistic of \textbf{approximately 4.91} means that the observed
difference in satisfaction proportions (between plant-based and
traditional options) is \textbf{4.91 standard errors above what we would
expect} under the assumption that there is \textbf{no true difference}
in satisfaction levels.

In context, this suggests that the observed satisfaction gap (72.5\%
vs.~60\%) is \textbf{much larger than what we would expect from random
variation alone}, providing \textbf{strong statistical evidence} that
Cal Poly students are \textbf{more satisfied with the plant-based
entrees} compared to the traditional menu.

\begin{center}\rule{0.5\linewidth}{0.5pt}\end{center}

\hypertarget{rubric-test-statistic-and-interpretation-total-4-points}{%
\subsection{\texorpdfstring{\textbf{Rubric: Test Statistic and
Interpretation (Total: 4
points)}}{Rubric: Test Statistic and Interpretation (Total: 4 points)}}\label{rubric-test-statistic-and-interpretation-total-4-points}}

\begin{longtable}[]{@{}
  >{\raggedright\arraybackslash}p{(\columnwidth - 4\tabcolsep) * \real{0.1918}}
  >{\raggedright\arraybackslash}p{(\columnwidth - 4\tabcolsep) * \real{0.6164}}
  >{\raggedright\arraybackslash}p{(\columnwidth - 4\tabcolsep) * \real{0.1918}}@{}}
\toprule\noalign{}
\begin{minipage}[b]{\linewidth}\raggedright
\textbf{Component}
\end{minipage} & \begin{minipage}[b]{\linewidth}\raggedright
\textbf{Criteria}
\end{minipage} & \begin{minipage}[b]{\linewidth}\raggedright
\textbf{Points}
\end{minipage} \\
\midrule\noalign{}
\endhead
\bottomrule\noalign{}
\endlastfoot
\textbf{Correct Formula Applied} & Correctly sets up the two-sample
z-test formula using ( \frac{\hat{p}_1 - \hat{p}_2}{SE} ) & 1.0 \\
\textbf{Correct Numerical Calculation} & Correctly calculates ( z
\approx 4.91 ), or shows accurate arithmetic using provided values &
1.0 \\
\textbf{General Interpretation of Z-Score} & Explains what the test
statistic means: how far the observed difference is from the null, in
standard errors & 1.0 \\
\textbf{Contextual Interpretation} & Connects z-score back to the
specific comparison between \textbf{plant-based vs.~traditional
satisfaction levels} among students & 1.0 \\
\end{longtable}



\end{document}
