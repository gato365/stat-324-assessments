% Options for packages loaded elsewhere
\PassOptionsToPackage{unicode}{hyperref}
\PassOptionsToPackage{hyphens}{url}
\PassOptionsToPackage{dvipsnames,svgnames,x11names}{xcolor}
%
\documentclass[
  letterpaper,
  DIV=11,
  numbers=noendperiod]{scrartcl}

\usepackage{amsmath,amssymb}
\usepackage{iftex}
\ifPDFTeX
  \usepackage[T1]{fontenc}
  \usepackage[utf8]{inputenc}
  \usepackage{textcomp} % provide euro and other symbols
\else % if luatex or xetex
  \usepackage{unicode-math}
  \defaultfontfeatures{Scale=MatchLowercase}
  \defaultfontfeatures[\rmfamily]{Ligatures=TeX,Scale=1}
\fi
\usepackage{lmodern}
\ifPDFTeX\else  
    % xetex/luatex font selection
\fi
% Use upquote if available, for straight quotes in verbatim environments
\IfFileExists{upquote.sty}{\usepackage{upquote}}{}
\IfFileExists{microtype.sty}{% use microtype if available
  \usepackage[]{microtype}
  \UseMicrotypeSet[protrusion]{basicmath} % disable protrusion for tt fonts
}{}
\makeatletter
\@ifundefined{KOMAClassName}{% if non-KOMA class
  \IfFileExists{parskip.sty}{%
    \usepackage{parskip}
  }{% else
    \setlength{\parindent}{0pt}
    \setlength{\parskip}{6pt plus 2pt minus 1pt}}
}{% if KOMA class
  \KOMAoptions{parskip=half}}
\makeatother
\usepackage{xcolor}
\setlength{\emergencystretch}{3em} % prevent overfull lines
\setcounter{secnumdepth}{-\maxdimen} % remove section numbering
% Make \paragraph and \subparagraph free-standing
\makeatletter
\ifx\paragraph\undefined\else
  \let\oldparagraph\paragraph
  \renewcommand{\paragraph}{
    \@ifstar
      \xxxParagraphStar
      \xxxParagraphNoStar
  }
  \newcommand{\xxxParagraphStar}[1]{\oldparagraph*{#1}\mbox{}}
  \newcommand{\xxxParagraphNoStar}[1]{\oldparagraph{#1}\mbox{}}
\fi
\ifx\subparagraph\undefined\else
  \let\oldsubparagraph\subparagraph
  \renewcommand{\subparagraph}{
    \@ifstar
      \xxxSubParagraphStar
      \xxxSubParagraphNoStar
  }
  \newcommand{\xxxSubParagraphStar}[1]{\oldsubparagraph*{#1}\mbox{}}
  \newcommand{\xxxSubParagraphNoStar}[1]{\oldsubparagraph{#1}\mbox{}}
\fi
\makeatother

\usepackage{color}
\usepackage{fancyvrb}
\newcommand{\VerbBar}{|}
\newcommand{\VERB}{\Verb[commandchars=\\\{\}]}
\DefineVerbatimEnvironment{Highlighting}{Verbatim}{commandchars=\\\{\}}
% Add ',fontsize=\small' for more characters per line
\usepackage{framed}
\definecolor{shadecolor}{RGB}{241,243,245}
\newenvironment{Shaded}{\begin{snugshade}}{\end{snugshade}}
\newcommand{\AlertTok}[1]{\textcolor[rgb]{0.68,0.00,0.00}{#1}}
\newcommand{\AnnotationTok}[1]{\textcolor[rgb]{0.37,0.37,0.37}{#1}}
\newcommand{\AttributeTok}[1]{\textcolor[rgb]{0.40,0.45,0.13}{#1}}
\newcommand{\BaseNTok}[1]{\textcolor[rgb]{0.68,0.00,0.00}{#1}}
\newcommand{\BuiltInTok}[1]{\textcolor[rgb]{0.00,0.23,0.31}{#1}}
\newcommand{\CharTok}[1]{\textcolor[rgb]{0.13,0.47,0.30}{#1}}
\newcommand{\CommentTok}[1]{\textcolor[rgb]{0.37,0.37,0.37}{#1}}
\newcommand{\CommentVarTok}[1]{\textcolor[rgb]{0.37,0.37,0.37}{\textit{#1}}}
\newcommand{\ConstantTok}[1]{\textcolor[rgb]{0.56,0.35,0.01}{#1}}
\newcommand{\ControlFlowTok}[1]{\textcolor[rgb]{0.00,0.23,0.31}{\textbf{#1}}}
\newcommand{\DataTypeTok}[1]{\textcolor[rgb]{0.68,0.00,0.00}{#1}}
\newcommand{\DecValTok}[1]{\textcolor[rgb]{0.68,0.00,0.00}{#1}}
\newcommand{\DocumentationTok}[1]{\textcolor[rgb]{0.37,0.37,0.37}{\textit{#1}}}
\newcommand{\ErrorTok}[1]{\textcolor[rgb]{0.68,0.00,0.00}{#1}}
\newcommand{\ExtensionTok}[1]{\textcolor[rgb]{0.00,0.23,0.31}{#1}}
\newcommand{\FloatTok}[1]{\textcolor[rgb]{0.68,0.00,0.00}{#1}}
\newcommand{\FunctionTok}[1]{\textcolor[rgb]{0.28,0.35,0.67}{#1}}
\newcommand{\ImportTok}[1]{\textcolor[rgb]{0.00,0.46,0.62}{#1}}
\newcommand{\InformationTok}[1]{\textcolor[rgb]{0.37,0.37,0.37}{#1}}
\newcommand{\KeywordTok}[1]{\textcolor[rgb]{0.00,0.23,0.31}{\textbf{#1}}}
\newcommand{\NormalTok}[1]{\textcolor[rgb]{0.00,0.23,0.31}{#1}}
\newcommand{\OperatorTok}[1]{\textcolor[rgb]{0.37,0.37,0.37}{#1}}
\newcommand{\OtherTok}[1]{\textcolor[rgb]{0.00,0.23,0.31}{#1}}
\newcommand{\PreprocessorTok}[1]{\textcolor[rgb]{0.68,0.00,0.00}{#1}}
\newcommand{\RegionMarkerTok}[1]{\textcolor[rgb]{0.00,0.23,0.31}{#1}}
\newcommand{\SpecialCharTok}[1]{\textcolor[rgb]{0.37,0.37,0.37}{#1}}
\newcommand{\SpecialStringTok}[1]{\textcolor[rgb]{0.13,0.47,0.30}{#1}}
\newcommand{\StringTok}[1]{\textcolor[rgb]{0.13,0.47,0.30}{#1}}
\newcommand{\VariableTok}[1]{\textcolor[rgb]{0.07,0.07,0.07}{#1}}
\newcommand{\VerbatimStringTok}[1]{\textcolor[rgb]{0.13,0.47,0.30}{#1}}
\newcommand{\WarningTok}[1]{\textcolor[rgb]{0.37,0.37,0.37}{\textit{#1}}}

\providecommand{\tightlist}{%
  \setlength{\itemsep}{0pt}\setlength{\parskip}{0pt}}\usepackage{longtable,booktabs,array}
\usepackage{calc} % for calculating minipage widths
% Correct order of tables after \paragraph or \subparagraph
\usepackage{etoolbox}
\makeatletter
\patchcmd\longtable{\par}{\if@noskipsec\mbox{}\fi\par}{}{}
\makeatother
% Allow footnotes in longtable head/foot
\IfFileExists{footnotehyper.sty}{\usepackage{footnotehyper}}{\usepackage{footnote}}
\makesavenoteenv{longtable}
\usepackage{graphicx}
\makeatletter
\def\maxwidth{\ifdim\Gin@nat@width>\linewidth\linewidth\else\Gin@nat@width\fi}
\def\maxheight{\ifdim\Gin@nat@height>\textheight\textheight\else\Gin@nat@height\fi}
\makeatother
% Scale images if necessary, so that they will not overflow the page
% margins by default, and it is still possible to overwrite the defaults
% using explicit options in \includegraphics[width, height, ...]{}
\setkeys{Gin}{width=\maxwidth,height=\maxheight,keepaspectratio}
% Set default figure placement to htbp
\makeatletter
\def\fps@figure{htbp}
\makeatother

\KOMAoption{captions}{tableheading}
\makeatletter
\@ifpackageloaded{caption}{}{\usepackage{caption}}
\AtBeginDocument{%
\ifdefined\contentsname
  \renewcommand*\contentsname{Table of contents}
\else
  \newcommand\contentsname{Table of contents}
\fi
\ifdefined\listfigurename
  \renewcommand*\listfigurename{List of Figures}
\else
  \newcommand\listfigurename{List of Figures}
\fi
\ifdefined\listtablename
  \renewcommand*\listtablename{List of Tables}
\else
  \newcommand\listtablename{List of Tables}
\fi
\ifdefined\figurename
  \renewcommand*\figurename{Figure}
\else
  \newcommand\figurename{Figure}
\fi
\ifdefined\tablename
  \renewcommand*\tablename{Table}
\else
  \newcommand\tablename{Table}
\fi
}
\@ifpackageloaded{float}{}{\usepackage{float}}
\floatstyle{ruled}
\@ifundefined{c@chapter}{\newfloat{codelisting}{h}{lop}}{\newfloat{codelisting}{h}{lop}[chapter]}
\floatname{codelisting}{Listing}
\newcommand*\listoflistings{\listof{codelisting}{List of Listings}}
\makeatother
\makeatletter
\makeatother
\makeatletter
\@ifpackageloaded{caption}{}{\usepackage{caption}}
\@ifpackageloaded{subcaption}{}{\usepackage{subcaption}}
\makeatother

\ifLuaTeX
  \usepackage{selnolig}  % disable illegal ligatures
\fi
\usepackage{bookmark}

\IfFileExists{xurl.sty}{\usepackage{xurl}}{} % add URL line breaks if available
\urlstyle{same} % disable monospaced font for URLs
\hypersetup{
  pdftitle={Exam 2 Solutions - STAT 324},
  colorlinks=true,
  linkcolor={blue},
  filecolor={Maroon},
  citecolor={Blue},
  urlcolor={Blue},
  pdfcreator={LaTeX via pandoc}}


\title{Exam 2 Solutions - STAT 324}
\author{}
\date{}

\begin{document}
\maketitle


\subsection{Question 4}\label{question-4}

\subsubsection{Question:}\label{question}

Based on the best model output and supporting diagnostics (e.g., VIF
values), is multicollinearity a problem in the model? Explain your
reasoning clearly.

\textbf{Total Points:} 15

\begin{center}\rule{0.5\linewidth}{0.5pt}\end{center}

\subsubsection{Solution:}\label{solution}

Based on the provided model output and VIF values, multicollinearity is
\textbf{not considered a significant problem} in this model.

Here's the reasoning:

\begin{enumerate}
\def\labelenumi{\arabic{enumi}.}
\item
  \textbf{Examination of VIF Values:} The Variance Inflation Factors
  (VIFs) for the predictor variables are as follows:

  \begin{itemize}
  \tightlist
  \item
    Agriculture: 2.147
  \item
    Education: 1.816
  \item
    Catholic: 1.299
  \item
    Infant.Mortality: 1.108
  \end{itemize}
\item
  \textbf{Interpretation of VIF Values:} VIF values quantify how much
  the variance of the estimate of a regression coefficient is increased
  due to multicollinearity. A commonly accepted guideline is that VIF
  values above 5 or 10 indicate problematic levels of multicollinearity.
\item
  \textbf{Conclusion:} All the calculated VIF values in this model are
  well below the common thresholds of 5 or 10. The highest VIF is
  approximately 2.15 for Agriculture, which is far from suggesting a
  significant issue with multicollinearity. Therefore, the predictor
  variables in this model do not appear to be excessively correlated
  with each other to a degree that would severely impact the reliability
  or interpretation of the coefficient estimates.
\end{enumerate}

\textbf{In summary, the low VIF values indicate that multicollinearity
is not a significant concern for the \texttt{best\_model\_swiss}.}

\begin{center}\rule{0.5\linewidth}{0.5pt}\end{center}

\subsubsection{Rubric:}\label{rubric}

\begin{longtable}[]{@{}
  >{\raggedright\arraybackslash}p{(\columnwidth - 4\tabcolsep) * \real{0.2340}}
  >{\raggedright\arraybackslash}p{(\columnwidth - 4\tabcolsep) * \real{0.0319}}
  >{\raggedright\arraybackslash}p{(\columnwidth - 4\tabcolsep) * \real{0.7340}}@{}}
\toprule\noalign{}
\begin{minipage}[b]{\linewidth}\raggedright
Criteria
\end{minipage} & \begin{minipage}[b]{\linewidth}\raggedright
Points
\end{minipage} & \begin{minipage}[b]{\linewidth}\raggedright
Descriptors
\end{minipage} \\
\midrule\noalign{}
\endhead
\bottomrule\noalign{}
\endlastfoot
\textbf{Identification of VIF Values} & 3 & Correctly identifies and
lists the VIF values for each predictor variable from the output. \\
\textbf{Understanding of Multicollinearity \& VIF} & 5 & Explains what
multicollinearity is and how VIF values are used to detect it. Mentions
common VIF thresholds (e.g., 5 or 10). \\
\textbf{Analysis of VIF Values} & 4 & Compares the observed VIF values
to the accepted thresholds. Correctly states whether the values indicate
a multicollinearity problem. \\
\textbf{Clear Explanation and Conclusion} & 3 & Provides a clear and
concise explanation of the reasoning. Draws a correct conclusion about
the presence or absence of problematic multicollinearity in the
model. \\
\textbf{Overall Clarity and Organization} & & (Implicitly assessed
throughout) Response is well-organized and easy to understand. \\
\end{longtable}

\newpage

\subsection{Question 8}\label{question-8}

\subsubsection{Question:}\label{question-1}

Plant scientists are investigating how the rate of photosynthesis (Y,
measured in mols) in a newly developed rice variety responds to
different light intensities (X, measured in lux). The experiments cover
a wide range of light intensities, from very low light (e.g., 50 lux) up
to moderately high levels (e.g., 15,000 lux), as rice photosynthesis
tends to saturate. They found that the model fits the observed data
extremely well across this broad spectrum of light. This model
effectively captures the initial steep rise in photosynthesis at low
light levels and the subsequent diminishing returns as light intensity
increases further, before other factors become limiting.

Suppose we want to predict the photosynthetic rate for this rice variety
at a light intensity of 7,500 lux. Show your work and explained your
answer while mention any assumptions you are making.

A fake R output for this model might look like:

\begin{Shaded}
\begin{Highlighting}[]
\CommentTok{\# Model: lm(formula = photosynthetic\_rate \textasciitilde{} log10(light\_intensity\_lux), data = rice\_photosynthesis\_study)}
\CommentTok{\# Coefficients:}
\CommentTok{\# (Intercept)                   log10(light\_intensity\_lux)}
\CommentTok{\# {-}8.500                                 7.200}
\CommentTok{\# {-}{-}{-}}
\CommentTok{\# Signif. codes:  0 ‘***’ 0.001 ‘**’ 0.01 ‘*’ 0.05 ‘.’ 0.1 ‘ ’ 1}
\CommentTok{\# {-}{-}{-}}
\CommentTok{\# Residual standard error: 1.85 on 78 degrees of freedom}
\CommentTok{\# Multiple R{-}squared: 0.844,    Adjusted R{-}squared: 0.842}
\CommentTok{\# F{-}statistic: 423.2 on 1 and 78 DF,  p{-}value: \textless{} 2.2e{-}16}
\end{Highlighting}
\end{Shaded}

\subsubsection{Solution:}\label{solution-1}

The model relates photosynthetic rate (\(Y\)) to the base-10 logarithm
of light intensity (\(X\)). Based on the R output, the estimated linear
model is:

\(Y = \beta_0 + \beta_1 \times \log_{10}(X)\)

where: - \(Y\) is the predicted photosynthetic rate - \(X\) is the light
intensity in lux - \(\beta_0\) is the intercept coefficient -
\(\beta_1\) is the coefficient for \(\log_{10}(X)\)

From the provided R output: - Estimated Intercept (\(\hat{\beta}_0\)):
-8.500 - Estimated coefficient for
\(\log_{10}(\text{light\_intensity\_lux})\) (\(\hat{\beta}_1\)): 7.200

We want to predict the photosynthetic rate when the light intensity
(\(X\)) is 7,500 lux.

First, calculate the base-10 logarithm of 7,500:
\(\log_{10}(7500) \approx 3.875\)

Now, substitute this value and the estimated coefficients into the model
equation: \(Y = -8.500 + 7.200 \times 3.875\) \(Y = -8.500 + 27.900\)
\(Y = 19.400\)

Therefore, the predicted photosynthetic rate at a light intensity of
7,500 lux is approximately 19.400 mols.

\textbf{Explanation:} The model uses the logarithm of light intensity
because the relationship between light and photosynthesis is often
non-linear, exhibiting diminishing returns at higher intensities. Taking
the log can linearize this relationship. By plugging the log of the
target light intensity (7,500 lux) into the estimated linear equation,
we obtain the predicted photosynthetic rate based on the relationship
learned from the data.

\textbf{Assumption:} A key assumption being made is that the linear
relationship between the photosynthetic rate and the log of light
intensity, as modeled, holds true and is generalizable to a light
intensity of 7,500 lux, which falls within or is close to the range of
light intensities the model was trained on (50 to 15,000 lux). We also
assume that other factors not included in this simple model (like CO2
concentration, temperature, etc.) are not limiting or are held constant
in a way consistent with the data used to build the model.

\subsubsection{Rubric:}\label{rubric-1}

\begin{longtable}[]{@{}
  >{\raggedright\arraybackslash}p{(\columnwidth - 4\tabcolsep) * \real{0.2840}}
  >{\raggedright\arraybackslash}p{(\columnwidth - 4\tabcolsep) * \real{0.0247}}
  >{\raggedright\arraybackslash}p{(\columnwidth - 4\tabcolsep) * \real{0.6914}}@{}}
\toprule\noalign{}
\begin{minipage}[b]{\linewidth}\raggedright
Criteria
\end{minipage} & \begin{minipage}[b]{\linewidth}\raggedright
Points
\end{minipage} & \begin{minipage}[b]{\linewidth}\raggedright
Descriptors
\end{minipage} \\
\midrule\noalign{}
\endhead
\bottomrule\noalign{}
\endlastfoot
\textbf{Identification of Model Structure} & 2 & Correctly identifies
the model as a linear relationship between photosynthetic rate and the
log10 of light intensity
(\(Y = \beta_0 + \beta_1 \times \log_{10}(X)\)). \\
\textbf{Extraction of Coefficients} & 2 & Correctly extracts both the
intercept and the log10 coefficient values from the provided R
output. \\
\textbf{Calculation of \(\log_{10}\)(7500)} & 3 & Correctly calculates
or approximates the base-10 logarithm of 7,500 lux. \\
\textbf{Setting up and Performing Prediction Calculation} & 5 &
Correctly substitutes the calculated log value and the coefficients into
the model equation and performs the calculation to find the predicted
photosynthetic rate. \\
\textbf{Showing Work} & 1 & Clearly shows the steps involved in the
calculation process. \\
\textbf{Explanation of Prediction} & 1 & Provides a clear explanation of
how the prediction is made and what the result represents. \\
\textbf{Stating Relevant Assumption(s)} & 1 & Mentions at least one
relevant assumption made when performing the prediction (e.g., model
validity within the range, other factors constant). \\
\end{longtable}

\subsection{Question 9}\label{question-9}

\subsubsection{Question:}\label{question-2}

Based on the previous model, the slope coefficient for
log\(_{10}\)(light\_intensity\_lux) is 7.2. What does this coefficient
tell you about how photosynthetic rate changes with increasing light
intensity? Explain using interpretation of a log\(_{10}\) transformation
in context.

\textbf{Total Points:} 15

\begin{center}\rule{0.5\linewidth}{0.5pt}\end{center}

\subsubsection{Solution:}\label{solution-2}

The model is given by: Photosynthetic Rate
\(= \beta_0 + \beta_1 \times \log_{10}(\text{Light Intensity})\)

From the previous question, the estimated slope coefficient
(\(\hat{\beta}_1\)) for \(\log_{10}(\text{light\_intensity\_lux})\) is
7.200.

The slope coefficient in a regression model with a log-transformed
predictor tells us about the change in the response variable associated
with a \emph{unit increase in the log-transformed predictor}. In the
case of a \(\log_{10}\) transformation, a one-unit increase in
\(\log_{10}(X)\) corresponds to a tenfold increase in the original
variable \(X\). This is because if
\(\log_{10}(X_2) = \log_{10}(X_1) + 1\), then
\(X_2 = 10^{\log_{10}(X_1) + 1} = 10^{\log_{10}(X_1)} \times 10^1 = X_1 \times 10\).

Therefore, the interpretation of the slope coefficient 7.200 in this
context is:

For every tenfold increase in light intensity (measured in lux), the
photosynthetic rate (measured in mols) is estimated to increase by 7.200
units, holding other factors constant (although no other factors are in
this simple model).

\textbf{Explanation using context:} The log\(_{10}\) transformation is
used here likely because the relationship between light intensity and
photosynthetic rate is not linear in its original scale; it typically
shows a rapid increase at low light levels and then plateaus or
saturates at higher levels. The log transformation helps to linearize
this relationship. The coefficient of 7.200 signifies the average change
in photosynthetic rate for a proportional increase in light intensity
(specifically, a 10-fold increase). This means that increasing light
intensity from, say, 100 lux to 1000 lux is associated with an estimated
7.200 mols increase in photosynthetic rate, just as increasing it from
500 lux to 5000 lux is associated with a similar estimated increase of
7.200 mols. This reflects the diminishing \emph{absolute} return as
light intensity increases, although the return per tenfold increase
remains constant in this log-linear model.

\begin{center}\rule{0.5\linewidth}{0.5pt}\end{center}

\subsubsection{Rubric:}\label{rubric-2}

\begin{longtable}[]{@{}
  >{\raggedright\arraybackslash}p{(\columnwidth - 4\tabcolsep) * \real{0.2711}}
  >{\raggedright\arraybackslash}p{(\columnwidth - 4\tabcolsep) * \real{0.0211}}
  >{\raggedright\arraybackslash}p{(\columnwidth - 4\tabcolsep) * \real{0.7077}}@{}}
\toprule\noalign{}
\begin{minipage}[b]{\linewidth}\raggedright
Criteria
\end{minipage} & \begin{minipage}[b]{\linewidth}\raggedright
Points
\end{minipage} & \begin{minipage}[b]{\linewidth}\raggedright
Descriptors
\end{minipage} \\
\midrule\noalign{}
\endhead
\bottomrule\noalign{}
\endlastfoot
\textbf{Identification of Slope Coefficient} & 2 & Correctly identifies
the slope coefficient value for
\(\log_{10}(\text{light\_intensity\_lux})\) as 7.2 or 7.200. \\
\textbf{Understanding of \(\log_{10}\) Interpretation} & 5 & Explains
the general principle of interpreting a slope when the predictor is
\(\log_{10}\) transformed -- relates a unit change in the log scale to a
tenfold change in the original scale. \\
\textbf{Contextual Interpretation (Photosynthetic Rate \& Light
Intensity)} & 4 & Applies the interpretation specifically to the
variables in the problem (photosynthetic rate and light intensity). \\
\textbf{Using the Specific Coefficient Value} & 3 & Includes the
specific value of the coefficient (7.200) in the interpretation
statement. \\
\textbf{Explanation of Log Transformation Role} & 1 & Briefly explains
why a log transformation might be used in this context (e.g., to handle
non-linearity or diminishing returns) or clarifies the meaning of the
coefficient in the context of the non-linear relationship. \\
\end{longtable}

\subsection{Question 10}\label{question-10}

\subsubsection{Question}\label{question-3}

A market analyst is studying the relationship between the number of
unique monthly visitors to e-commerce websites (X) and their total
monthly sales revenue (Y, in US dollars). The dataset includes a diverse
range of websites, from small niche stores with a few thousand visitors
(e.g., 1,000) to larger platforms attracting up to 1,000,000 or more
visitors per month. The analyst found that a log-log model that provides
a very strong and interpretable fit. This model suggests that a
percentage change in website traffic tends to correspond to a relatively
consistent percentage change in sales revenue across websites of
different sizes.

Suppose we want to predict the monthly sales revenue for a website that
receives 90,000 unique monthly visitors. Show your work and explained
your answer while mention any assumptions you are making.

A fake R output for this model might look like:

\begin{Shaded}
\begin{Highlighting}[]
\CommentTok{\# Model: lm(formula = log10(monthly\_sales\_usd) \textasciitilde{} log10(unique\_monthly\_visitors), data = ecommerce\_traffic\_sales)}
\CommentTok{\# Coefficients:}
\CommentTok{\# (Intercept)                   log10(unique\_monthly\_visitors)}
\CommentTok{\# 0.1500                                 1.1200}
\CommentTok{\# {-}{-}{-}}
\CommentTok{\# Signif. codes:  0 ‘***’ 0.001 ‘**’ 0.01 ‘*’ 0.05 ‘.’ 0.1 ‘ ’ 1}
\CommentTok{\# {-}{-}{-}}
\CommentTok{\# Residual standard error: 0.18 on 148 degrees of freedom}
\CommentTok{\# Multiple R{-}squared: 0.950,    Adjusted R{-}squared: 0.950}
\CommentTok{\# F{-}statistic: 2844 on 1 and 148 DF,  p{-}value: \textless{} 2.2e{-}16}
\end{Highlighting}
\end{Shaded}

\textbf{Total Points:} 15

\begin{center}\rule{0.5\linewidth}{0.5pt}\end{center}

\subsubsection{Solution:}\label{solution-3}

The model is \(\log_{10}(Y) = \beta_0 + \beta_1 \times \log_{10}(X)\),
where \(Y\) is sales revenue and \(X\) is unique visitors. From the R
output, the estimated coefficients are \(\hat{\beta}_0 = 0.1500\) and
\(\hat{\beta}_1 = 1.1200\).

We need to predict \(Y\) for \(X = 90,000\).

\begin{enumerate}
\def\labelenumi{\arabic{enumi}.}
\item
  Calculate \(\log_{10}(90000)\): \(\log_{10}(90000) \approx 4.9542\)
\item
  Calculate the predicted \(\log_{10}(Y)\):
  \(\log_{10}(\hat{Y}) = 0.1500 + 1.1200 \times 4.9542\)
  \(\log_{10}(\hat{Y}) = 0.1500 + 5.5487\)
  \(\log_{10}(\hat{Y}) = 5.6987\)
\item
  Exponentiate to find the predicted \(Y\):
  \(\hat{Y} = 10^{5.6987} \approx 500000\)
\end{enumerate}

The predicted monthly sales revenue for a website with 90,000 unique
monthly visitors is approximately \$500,000. The log-log model indicates
a proportional relationship; here, a 1\% increase in visitors is
associated with an estimated 1.12\% increase in sales.

\textbf{Assumption:} The model is valid for predicting at 90,000
visitors, and other factors influencing sales are consistent with the
model's data.

\begin{center}\rule{0.5\linewidth}{0.5pt}\end{center}

\subsubsection{Rubric:}\label{rubric-3}

\begin{longtable}[]{@{}
  >{\raggedright\arraybackslash}p{(\columnwidth - 4\tabcolsep) * \real{0.3224}}
  >{\raggedright\arraybackslash}p{(\columnwidth - 4\tabcolsep) * \real{0.0280}}
  >{\raggedright\arraybackslash}p{(\columnwidth - 4\tabcolsep) * \real{0.6495}}@{}}
\toprule\noalign{}
\begin{minipage}[b]{\linewidth}\raggedright
Criteria
\end{minipage} & \begin{minipage}[b]{\linewidth}\raggedright
Points
\end{minipage} & \begin{minipage}[b]{\linewidth}\raggedright
Descriptors
\end{minipage} \\
\midrule\noalign{}
\endhead
\bottomrule\noalign{}
\endlastfoot
\textbf{Identify Model \& Coefficients} & 4 & Correctly identifies the
log-log model structure (\(\log_{10}(Y) \sim \log_{10}(X)\)) and
extracts the intercept (0.1500) and slope (1.1200) coefficients. \\
\textbf{Calculate \(\log_{10}\)(90000)} & 3 & Correctly calculates or
approximates \(\log_{10}(90000)\). \\
\textbf{Predict \(\log_{10}\)(Y)} & 4 & Correctly sets up and calculates
the predicted value of \(\log_{10}(Y)\) using the extracted coefficients
and the calculated log visitor count. \\
\textbf{Exponentiate to find Predicted Y} & 3 & Correctly performs the
base-10 exponentiation to convert the predicted \(\log_{10}(Y)\) back to
the original sales revenue scale (\(Y\)). \\
\textbf{Show Work \& State Assumption} & 1 & Clearly shows the main
calculation steps and states a relevant assumption for the
prediction. \\
\end{longtable}




\end{document}
