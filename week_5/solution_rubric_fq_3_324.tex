% Options for packages loaded elsewhere
\PassOptionsToPackage{unicode}{hyperref}
\PassOptionsToPackage{hyphens}{url}
\PassOptionsToPackage{dvipsnames,svgnames,x11names}{xcolor}
%
\documentclass[
  letterpaper,
  DIV=11,
  numbers=noendperiod]{scrartcl}

\usepackage{amsmath,amssymb}
\usepackage{iftex}
\ifPDFTeX
  \usepackage[T1]{fontenc}
  \usepackage[utf8]{inputenc}
  \usepackage{textcomp} % provide euro and other symbols
\else % if luatex or xetex
  \usepackage{unicode-math}
  \defaultfontfeatures{Scale=MatchLowercase}
  \defaultfontfeatures[\rmfamily]{Ligatures=TeX,Scale=1}
\fi
\usepackage{lmodern}
\ifPDFTeX\else  
    % xetex/luatex font selection
\fi
% Use upquote if available, for straight quotes in verbatim environments
\IfFileExists{upquote.sty}{\usepackage{upquote}}{}
\IfFileExists{microtype.sty}{% use microtype if available
  \usepackage[]{microtype}
  \UseMicrotypeSet[protrusion]{basicmath} % disable protrusion for tt fonts
}{}
\makeatletter
\@ifundefined{KOMAClassName}{% if non-KOMA class
  \IfFileExists{parskip.sty}{%
    \usepackage{parskip}
  }{% else
    \setlength{\parindent}{0pt}
    \setlength{\parskip}{6pt plus 2pt minus 1pt}}
}{% if KOMA class
  \KOMAoptions{parskip=half}}
\makeatother
\usepackage{xcolor}
\setlength{\emergencystretch}{3em} % prevent overfull lines
\setcounter{secnumdepth}{-\maxdimen} % remove section numbering
% Make \paragraph and \subparagraph free-standing
\makeatletter
\ifx\paragraph\undefined\else
  \let\oldparagraph\paragraph
  \renewcommand{\paragraph}{
    \@ifstar
      \xxxParagraphStar
      \xxxParagraphNoStar
  }
  \newcommand{\xxxParagraphStar}[1]{\oldparagraph*{#1}\mbox{}}
  \newcommand{\xxxParagraphNoStar}[1]{\oldparagraph{#1}\mbox{}}
\fi
\ifx\subparagraph\undefined\else
  \let\oldsubparagraph\subparagraph
  \renewcommand{\subparagraph}{
    \@ifstar
      \xxxSubParagraphStar
      \xxxSubParagraphNoStar
  }
  \newcommand{\xxxSubParagraphStar}[1]{\oldsubparagraph*{#1}\mbox{}}
  \newcommand{\xxxSubParagraphNoStar}[1]{\oldsubparagraph{#1}\mbox{}}
\fi
\makeatother


\providecommand{\tightlist}{%
  \setlength{\itemsep}{0pt}\setlength{\parskip}{0pt}}\usepackage{longtable,booktabs,array}
\usepackage{calc} % for calculating minipage widths
% Correct order of tables after \paragraph or \subparagraph
\usepackage{etoolbox}
\makeatletter
\patchcmd\longtable{\par}{\if@noskipsec\mbox{}\fi\par}{}{}
\makeatother
% Allow footnotes in longtable head/foot
\IfFileExists{footnotehyper.sty}{\usepackage{footnotehyper}}{\usepackage{footnote}}
\makesavenoteenv{longtable}
\usepackage{graphicx}
\makeatletter
\def\maxwidth{\ifdim\Gin@nat@width>\linewidth\linewidth\else\Gin@nat@width\fi}
\def\maxheight{\ifdim\Gin@nat@height>\textheight\textheight\else\Gin@nat@height\fi}
\makeatother
% Scale images if necessary, so that they will not overflow the page
% margins by default, and it is still possible to overwrite the defaults
% using explicit options in \includegraphics[width, height, ...]{}
\setkeys{Gin}{width=\maxwidth,height=\maxheight,keepaspectratio}
% Set default figure placement to htbp
\makeatletter
\def\fps@figure{htbp}
\makeatother

\KOMAoption{captions}{tableheading}
\makeatletter
\@ifpackageloaded{caption}{}{\usepackage{caption}}
\AtBeginDocument{%
\ifdefined\contentsname
  \renewcommand*\contentsname{Table of contents}
\else
  \newcommand\contentsname{Table of contents}
\fi
\ifdefined\listfigurename
  \renewcommand*\listfigurename{List of Figures}
\else
  \newcommand\listfigurename{List of Figures}
\fi
\ifdefined\listtablename
  \renewcommand*\listtablename{List of Tables}
\else
  \newcommand\listtablename{List of Tables}
\fi
\ifdefined\figurename
  \renewcommand*\figurename{Figure}
\else
  \newcommand\figurename{Figure}
\fi
\ifdefined\tablename
  \renewcommand*\tablename{Table}
\else
  \newcommand\tablename{Table}
\fi
}
\@ifpackageloaded{float}{}{\usepackage{float}}
\floatstyle{ruled}
\@ifundefined{c@chapter}{\newfloat{codelisting}{h}{lop}}{\newfloat{codelisting}{h}{lop}[chapter]}
\floatname{codelisting}{Listing}
\newcommand*\listoflistings{\listof{codelisting}{List of Listings}}
\makeatother
\makeatletter
\makeatother
\makeatletter
\@ifpackageloaded{caption}{}{\usepackage{caption}}
\@ifpackageloaded{subcaption}{}{\usepackage{subcaption}}
\makeatother

\ifLuaTeX
  \usepackage{selnolig}  % disable illegal ligatures
\fi
\usepackage{bookmark}

\IfFileExists{xurl.sty}{\usepackage{xurl}}{} % add URL line breaks if available
\urlstyle{same} % disable monospaced font for URLs
\hypersetup{
  pdftitle={Friday Quiz 4 Solutions - STAT 324},
  colorlinks=true,
  linkcolor={blue},
  filecolor={Maroon},
  citecolor={Blue},
  urlcolor={Blue},
  pdfcreator={LaTeX via pandoc}}


\title{Friday Quiz 4 Solutions - STAT 324}
\author{}
\date{}

\begin{document}
\maketitle


Here are \textbf{model solutions and rubrics} for your three open-ended
questions --- aligned with your STAT 324 objectives and ready for
grading in Canvas:

\begin{center}\rule{0.5\linewidth}{0.5pt}\end{center}

\subsubsection{\texorpdfstring{✅ Q8 -- \emph{Interpreting Coefficients
Post-Transformation}}{✅ Q8 -- Interpreting Coefficients Post-Transformation}}\label{q8-interpreting-coefficients-post-transformation}

\textbf{Model Answer (Full Credit -- 3/3):}

The slope of 0.004 means that for every 1-unit increase in RPM, the
log₁₀ of vibration increases by 0.004. Since the model is in log₁₀(y),
this means vibration intensity increases \textbf{multiplicatively} ---
specifically, by a factor of 10\^{}0.004 ≈ 1.0092 (about 0.92\%) for
each additional RPM. The intercept of --0.75 means that when RPM = 0,
the predicted log₁₀(vibration) is --0.75, which corresponds to a
vibration intensity of about 0.18 m/s². However, this is outside the
range of typical RPMs and should not be interpreted literally.

\textbf{Rubric:}

\begin{longtable}[]{@{}
  >{\raggedright\arraybackslash}p{(\columnwidth - 2\tabcolsep) * \real{0.9259}}
  >{\raggedright\arraybackslash}p{(\columnwidth - 2\tabcolsep) * \real{0.0741}}@{}}
\toprule\noalign{}
\begin{minipage}[b]{\linewidth}\raggedright
Component
\end{minipage} & \begin{minipage}[b]{\linewidth}\raggedright
Points
\end{minipage} \\
\midrule\noalign{}
\endhead
\bottomrule\noalign{}
\endlastfoot
Correct interpretation of slope (multiplicative change per RPM) & 1 \\
Clear back-transformation from log₁₀ to original scale & 1 \\
Reasonable interpretation of intercept with acknowledgment of
extrapolation & 1 \\
\end{longtable}

\begin{center}\rule{0.5\linewidth}{0.5pt}\end{center}

\subsubsection{\texorpdfstring{✅ Q9 -- \emph{Prediction at 1500
RPM}}{✅ Q9 -- Prediction at 1500 RPM}}\label{q9-prediction-at-1500-rpm}

\textbf{Model Answer (Full Credit -- 3/3):}

We are given the model: \textbf{log₁₀(vibration) = --0.75 + 0.004 × RPM}

Substituting RPM = 1500: log₁₀(vibration) = --0.75 + 0.004 × 1500 =
--0.75 + 6 = \textbf{5.25}

Now back-transform to the original scale: vibration = 10\^{}5.25 ≈
\textbf{177827.94 m/s²}

So, a machine running at 1500 RPM is predicted to have a vibration
intensity of about \textbf{177,828 m/s²}.

\textbf{Rubric:}

\begin{longtable}[]{@{}ll@{}}
\toprule\noalign{}
Component & Points \\
\midrule\noalign{}
\endhead
\bottomrule\noalign{}
\endlastfoot
Correct substitution and linear calculation of log₁₀(y) & 1 \\
Correct back-transformation using exponentiation & 1 \\
Final answer with units and context & 1 \\
\end{longtable}

\begin{center}\rule{0.5\linewidth}{0.5pt}\end{center}

\subsubsection{\texorpdfstring{✅ Q10 -- \emph{Interpreting the Impact
of
WLS}}{✅ Q10 -- Interpreting the Impact of WLS}}\label{q10-interpreting-the-impact-of-wls}

\textbf{Model Answer (Full Credit -- 3/3):}

\begin{enumerate}
\def\labelenumi{\arabic{enumi}.}
\item
  \textbf{Why WLS was needed}: The residual vs.~fitted plot showed a fan
  shape, suggesting increasing variance of residuals --- a violation of
  the constant variance (homoscedasticity) assumption. WLS was used to
  give less weight to points with higher residual variance.
\item
  \textbf{Effect on estimation}: WLS minimizes the \textbf{weighted} sum
  of squared residuals instead of treating all points equally. This
  adjusts the regression line to better fit the \textbf{more reliable
  points}, reducing the influence of high-variance observations.
\item
  \textbf{Impact on assumptions}: WLS helps correct heteroscedasticity.
  After applying WLS, the residual spread appeared more uniform across
  fitted values, making assumption checks more valid and improving
  inference accuracy.
\end{enumerate}

\textbf{Rubric:}

\begin{longtable}[]{@{}
  >{\raggedright\arraybackslash}p{(\columnwidth - 2\tabcolsep) * \real{0.9104}}
  >{\raggedright\arraybackslash}p{(\columnwidth - 2\tabcolsep) * \real{0.0896}}@{}}
\toprule\noalign{}
\begin{minipage}[b]{\linewidth}\raggedright
Component
\end{minipage} & \begin{minipage}[b]{\linewidth}\raggedright
Points
\end{minipage} \\
\midrule\noalign{}
\endhead
\bottomrule\noalign{}
\endlastfoot
Clear explanation of why OLS failed and WLS was needed & 1 \\
Accurate description of how WLS adjusts the fitting process & 1 \\
Discussion of improved assumption checks or plot improvements & 1 \\
\end{longtable}

\begin{center}\rule{0.5\linewidth}{0.5pt}\end{center}

Let me know if you want to export this to CSV or JSON for Canvas import!




\end{document}
